\documentclass{article}

\title{The SWL Thread System}
\author{John B. Zuckerman}

%
% Adapted from macros for R^nRS by John B. Zuckerman
%

\makeatletter

% Chapters, sections, etc.

\newcommand{\extrapart}[1]{
  \chapter*{#1}
  \markboth{#1}{#1}
  \vskip 1ex
  \addcontentsline{toc}{chapter}{#1}}

\newcommand{\clearchapterstar}[1]{
  \clearpage
  \topnewpage[
    \centerline{\large\bf\uppercase{#1}}
    \bigskip]}

\newcommand{\clearextrapart}[1]{
  \clearchapterstar{#1}
  \markboth{#1}{#1}
  \addcontentsline{toc}{chapter}{#1}}



\newcommand{\vest}{}
\newcommand{\dotsfoo}{$\ldots\,$}

\newcommand{\sharpfoo}[1]{{\tt\##1}}
\newcommand{\schfalse}{\sharpfoo{f}}
\newcommand{\schtrue}{\sharpfoo{t}}

\newcommand{\singlequote}{{\tt'}}  %\char19
\newcommand{\doublequote}{{\tt"}}
\newcommand{\backquote}{{\tt\char18}}
\newcommand{\backwhack}{{\tt\char`\\}}
\newcommand{\atsign}{{\tt\char`\@}}
\newcommand{\sharpsign}{{\tt\#}}
\newcommand{\verticalbar}{{\tt|}}

\newcommand{\coerce}{\discretionary{->}{}{->}}
\newcommand{\okbrk}{\discretionary{}{}{}}   % ok to break without inserting hyphen

\def\elem{\hbox{\raise.13ex\hbox{$\scriptstyle\in$}}}

\newcommand{\meta}[1]{{\noindent\hbox{\rm$\langle$#1$\rangle$}}}
\let\hyper=\meta

\newcommand{\lambdaexp}{lambda expression}
\newcommand{\Lambdaexp}{Lambda expression}
\newcommand{\cname}[1]{{\sf $<$#1$>$}}

%------------------------------------------------------------------
%
% mysterious stuff from the r4rs.
%
% \frobq will make quote and backquote look nicer.
\def\frobqcats{%\catcode`\'=13
\catcode`\`=13{}}
{\frobqcats
\gdef\frobqdefs{%\def'{\singlequote}
\def`{\backquote}}}
\def\frobq{\frobqcats\frobqdefs}

% \cf = code font
% Unfortunately, \cf \cf won't work at all, so don't even attempt to
% next constructions which use them...
\newcommand{\cf}{\frenchspacing\frobq\tt}

% Same as \obeycr, but doesn't do a \@gobblecr.
{\catcode`\^^M=13 \gdef\myobeycr{\catcode`\^^M=13 \def^^M{\\}}%
\gdef\restorecr{\catcode`\^^M=5 }}

{\catcode`\^^I=13 \gdef\obeytabs{\catcode`\^^I=13 \def^^I{\hbox{\hskip 4em}}}}

{\obeyspaces\gdef {\hbox{\hskip0.5em}}}

\gdef\gobblecr{\@gobblecr}

\def\setupcode{\@makeother\^}

%------------------------------------------------------------------

% Scheme example environment

\newenvironment{schemenoindent}{
  % Commands for scheme examples
  \newcommand{\ev}{\>\>\>\evalsto}       % three tabs
  \newcommand{\wev}{\>\>\>\>\evalsto}    % four tabs
  \newcommand{\wwev}{\>\>\>\>\>\evalsto} % five tabs
  \newcommand{\lev}{\\\>\evalsto}        % one tab
  \newcommand{\unspecified}{{\em{}unspecified}}
  \newcommand{\scherror}{{\em{}error}}
  \setupcode
  \small \cf \obeytabs \obeyspaces \myobeycr
  \begin{tabbing}%
\qquad \= \hspace*{4.5em} \= \hspace*{4.5em} \= \hspace*{4.5em} \= \hspace*{4.5em} \= \hspace*{4.5em} \= \kill
\gobblecr}{\end{tabbing}\unskip}


\newenvironment{scheme}{
  % Commands for scheme examples
  \newcommand{\ev}{\>\>\>\evalsto}       % three tabs
  \newcommand{\wev}{\>\>\>\>\evalsto}    % four tabs
  \newcommand{\wwev}{\>\>\>\>\>\evalsto} % five tabs
  \newcommand{\lev}{\\\>\evalsto}        % one tab
  \newcommand{\unspecified}{{\em{}unspecified}}
  \newcommand{\scherror}{{\em{}error}}
  \setupcode
  \small \cf \obeyspaces \myobeycr
  \begin{tabbing}%
\qquad \= \hspace*{4.5em} \= \hspace*{4.5em} \= \hspace*{4.5em} \= \hspace*{4.5em} \= \hspace*{4.5em} \= \+ \kill
\gobblecr}{\end{tabbing}\unskip}

\newcommand{\evalsto}{$\Longrightarrow$}

%
% The Proc Environment - arguments are:
%
% 1 - proc name 
% 2 - formals list
% 3 - return formal
%
\newenvironment{proc}[3]
   {\bigskip\goodbreak
   \noindent{\sf (#1{\sl #2})} \hfill {Scheme Procedure}\par{\indent
   Returns: \ifthenelse{\equal{#3}{}}{\em unspecified}{\em #3}
   }\par
   \begin{list}{}{
   \setlength{\leftmargin}{3ex}
   \setlength{\listparindent}{0pt}
   }\item[] 
   \penalty 10000
   }{
   \end{list}
   }


%
% The Method Environment - arguments are:
%
% 1 - proc name 
% 2 - formals list
% 3 - return formal
%
\newenvironment{method}[3]
   {\medskip\goodbreak
   \noindent{\sf (#1{\sl #2})} \hfill {Method}\par{\indent
   Returns:    \ifthenelse{\equal{#3}{}}{\em unspecified}{\em #3}
   }\par
   \begin{list}{}{
   \setlength{\leftmargin}{3ex}
   \setlength{\listparindent}{0pt}
   }\item[] 
   \penalty 10000
   }{
   \end{list}
   }


%
% The Class Environment - arguments are:
%
% 1 - class name
% 2 - parent name
% 3 - class type
%
\newenvironment{class}[3]
   {\bigskip\bigskip\bigskip\bigskip\goodbreak
    \noindent{\sf \cname{#1}}
    \ifthenelse{\equal{#2}{}}{}{ (parent: \cname{#2})} 
     \hfill {#3 Class}
   \par
   \begin{list}{}{
   \setlength{\leftmargin}{3ex}
   \setlength{\listparindent}{0pt}
   }\item[] 
   \penalty 10000
   }{
   \end{list}
   }





\begin{document}

\maketitle


%-* Introduction

\section{Introduction}

For several years the Programming Language Laboratory of the
Department of Computer Science at Indiana University has been
developing a graphical user interface extension to Scheme named {\sc swl}
(Scheme Widget Language), an integration of the Chez Scheme
implementation of Scheme\cite{TSPL}\cite{CSSM} and the interactive
graphics user interface capabilities of Tcl/Tk\cite{TCLTK}, plus
additional graphical capabilities hosted entirely in Scheme.  This
report documents the design and implementation of a major component of
{\sc swl}: the thread system.  Threads extend standard Scheme to provide
explicit concurrency.

Although it is possible to simulate the concurrent execution of
multiple independent graphical applications using a single thread of
control, primarily by exploiting various locality properties (e.g.,
keyboard input is typically associated with only one display window at
a time), a variety of user interface problems are more elegantly
solved by providing the Scheme application writer and the system
implementer with access to multiple independent threads of
control. Judicious use of multiple threads of control improves the
modularity of both the {\sc swl} system and many of its applications.

We have had several broad goals for this work:

\begin{itemize}

\item the thread system should be as transparent to the user, i.e.,
the application developer, as possible.  Ideally, the system should
retain a single-threaded ``look and feel.''

\item non-threaded Scheme programs should map into the threaded system
in a logical way,, i.e., any ordinary program that executes correctly
in standard Scheme should execute correctly as a single {\sc swl} thread.

\item the impact of thread system overhead on single-thread performance
should be minimal.

\item compatibility with Chez Scheme language extensions should be
maintained wherever possible. (Presently, however, the Chez Scheme
syntactic extension {\sf fluid-let} does not function with multiple
threads as might be expected; this is fully discussed below.)

\end{itemize}

The Chez Scheme implementation, plus certain ``system mode''
primitives that are available in Chez Scheme but which are not a part
of the documented Chez Scheme user interface, provide a sufficient
basis for implementing a thread system entirely in Scheme.  The Chez
Scheme notions of interrupt handling and timed preemption provide a
mechanism for interrupting execution of a Scheme program. Scheme's
general purpose control primitive {\sf call-with-current-continuation}
({\sf call/cc}) provides a means for defining new dynamic execution
contexts, or {\em continuations}, and for the transfer of control from
one continuation to another.  When augmented with appropriate
additional information, a Scheme continuation becomes a {\em thread},
or independently schedulable continuation.  The thread system provides
for the creation and destruction, suspension and resumption,
synchronization, and inspection of multiple threads of control.

Although we have tolerated some dependence on Chez Scheme internal
details, i.e., use of certain Chez Scheme system mode primitives,
our thread system requires no modifications to Chez Scheme at the
source code level.  However, we will also discuss potential extensions
to Chez Scheme that may be desirable to more efficiently support the
implementation of threads.


%-* Language Elements for Concurrency

\section{Language Elements for Concurrency}


In this section we introduce the basic concepts used in the remainder
of the report.  When considering the sequential flow of control within
an executing program, the current continuation is said to represent
the current dynamic execution context, the current stack of procedure
activation records, or simply ``the rest of the computation.''  In
Scheme, the primitive procedure {\sf call-with-current-continuation}
(or {\sf call/cc}) captures the current continuation and immediately
returns it to the running program as a first-class object.  A
continuation object is a procedure that, when invoked with zero or
more arguments, causes a transfer of control (called a continuation
``throw'') to the point where the continuation was captured.  The
arguments passed to the continuation object become the values returned
by the call to {\sf call/cc}.

The large majority of hardware processors presently available are
uniprocessors.  Unlike a parallel processor machine, a uniprocessor is
capable of executing only one machine instruction stream at a time.
However, using the familiar technique of time multiplexing (``time
slicing'') a uniprocessor may appear to perform many tasks at once by
executing multiple independent programs in an interleaved fashion.
The technique is to execute one program's instruction stream for a
(usually brief) period of time and, when the time allotted has
expired, to suspend that program and start another.  This approach to
concurrency known as {\em multitasking\/} or {\em multiprogramming},
and it is the approach that we take here.  We do not address
concurrency in a multiprocessor environment.

{\sf call/cc} captures but does not suspend the current continuation.
A Scheme program running in a system extended for concurrency,
however, may request its own suspension, i.e., temporarily give up
control to another Scheme program, by calling a system primitive that
we will name {\sf yield}. {\sf Yield} (1) captures the running
program's current continuation and stores it in a data structure, and
(2) transfers control to another continuation previously captured by a
call to {\sf yield}. The other continuation may or may not represent a
separate and unrelated computation.  In either case, in this report we
will call the data structure modified by {\sf yield} that stores the
continuation a {\em thread}, and we will term self-requested
suspension via {\sf yield} {\em non-preemptive multitasking}.  More
precisely, we define a thread to be a suspended Scheme continuation
object that is unique to the thread stored together with certain
additional data.  A running thread is a special case; it is not
suspended, and the meaning of its stored continuation is undefined.
In our uniprocessor architecture, there is at most one running thread
at any moment (however, many threads may be ready to run). When the
running thread invokes another thread's continuation by calling {\sf
yield}, we say that it {\em starts\/} or {\em runs\/} the other
thread.  Threads that cooperatively exchange control in this way are
also known as {\em co-routines}.

We have described how an existing thread may relinquish control to
another thread, but we have not yet described how to create new
threads. Therefore, we introduce a new primitive, {\sf fork}, which
takes a thunk (i.e., a Scheme procedure of no arguments) and creates a
new thread from it. {\sf fork} stores the new thread in a system data
structure and immediately returns an unspecified value to the caller.
Since by definition each thread contains a unique continuation object,
{\sf fork} must invoke {\sf call/cc} in order to obtain the new
thread's continuation.  Since {\sf fork} immediately returns an
unspecified value to the caller, we do not thereby obtain a mechanism
for thread synchronization or communication.  Therefore, in contrast
to some other parallel or concurrent systems, threads provide {\em
independent concurrency}.  For the moment, however, we may assume that
threads may communicate by means of assignment to shared
variables. Later, we will introduce appropriate synchronization and
communication primitives.

An executing Scheme program always has a continuation, or ``rest of
the computation.''  In a non-interactive implementation, control may
be transferred to a toplevel Scheme procedure that represents a
program by the operating system. In this case control should return to
the operating system when the Scheme program ``terminates'' by
returning to its caller.  However, Chez Scheme and many other systems
provide an {\em interactive\/} user interface.  That is, they contain
a program interpreter known as a {\em read-eval-print loop\/}, or 
{\sc repl}.  A {\sc repl} reads a Scheme program from an input device
(typically a keyboard), evaluates it, and prints the returned value on
an output device (typically a screen or window).  The {\sc repl} then
loops, waiting for another input. Generally, a {\sc repl} never
terminates on its own accord.

In a conventional non-threaded Scheme system, one may consider a {\sc repl}
to be a part of the external non-Scheme environment or even as a part
of the operating system, whether or not the {\sc repl} is actually
implemented in Scheme.  However, it is possible (and customary) to
implement a {\sc repl} as a Scheme procedure.  Here we will exploit the view
that a {\sc repl} is just an ordinary Scheme procedure that will execute
within its own thread.

While a {\sc repl} is waiting for input it generally has nothing to compute,
so it may relinquish control until the next input expression becomes
available.  In a non-threaded system, a {\sc repl} typically does this by
calling the Scheme primitive {\sf read}. If input is immediately
available, read returns it, otherwise read informs the underlying
operating system that the Scheme task is willing to block until input
becomes available.  This frees the operating system to suspend
execution of the Scheme task so that other waiting tasks may be
performed.  Once the operating system determines that {\sc repl} input has
become available, it will once again schedule the Scheme task for
execution.

In our multi-threaded design, the underlying operating system is not
informed that a given thread is willing to suspend.  Instead as
defined above, a suspending thread relinquishes control by calling the
thread system primitive {\sf yield}, and it is {\sf yield}'s
obligation to suspend the current thread and start another.  Thus, the
scheduling of Scheme threads is handled entirely within Scheme, and
the underlying operating system schedules only unrelated non-Scheme
tasks.  From the perspective of the operating system, the running
Scheme system itself represents a single task, and the operating
system has no knowledge of the multi-threading that is taking place
within Scheme.

A {\sc repl} is one kind of non-terminating program that handles external
requests and runs in its own thread, but there are others.  We will
call the class of all such programs {\em servers}.  Another kind of
server that is often found in interactive graphical applications is an
{\em event loop}.

Not all threads are servers, though, and some threads compute only for
a finite length of time.  What happens when a thread has completed its
computation?  Since we have defined a thread to be an independently
schedulable continuation, there is no parent continuation to which
control should be returned.  Instead, we arrange for the thread to
``terminate,'' by permanently suspending its execution.  A thread may
terminate by passing an appropriate flag value to {\sf yield} that
indicates the thread should be suspended and never restarted.

The type of concurrency we have just described is sometimes called
{\em non-preemptive multitasking}. That is, an executing thread
retains control until it decides to relinquish it by explicitly
calling {\sf yield}.  However, there are situations in which the
system should force the interruption of the current thread.  Such
situations also arise in conventional, non-threaded systems, e.g.,
when garbage collection becomes necessary, or when the user interrupts
a running program.  A thread system benefits from another kind of
interrupt mechanism known as {\em timed preemption}.  In this case a
running thread is allotted a time quantum, typically a fraction of a
second, and the system keeps track of how much of the quantum the
thread has consumed.  If the thread completes its computation before
the time quantum has been exhausted, then no action is taken by the
system.  However, if the thread exhausts its quantum before
terminating, the system suspends the thread by forcing a call to {\sf
yield}.

In Chez Scheme a running program may be allocated a time quantum in
the form of ``timer ticks.''  When the current thread's timer ticks
are exhausted the thread is interrupted and control is transferred to
the timer interrupt handler.  The timer interrupt handler is a Scheme
procedure that is defined by the thread system, but which executes in
the context of the currently running thread.  If the handler
determines that the thread has exceeded its quantum it will call {\sf
yield}, which effects the thread's suspension just as if the thread's
own code had called {\sf yield} explicitly.

It is sometimes necessary to protect certain regions of an executing
program from interruption.  Consider a program fragment that
increments a global variable.  To do this the program will: 

\begin{itemize}

\item read
the current value stored in the variable, 

\item increment the value by
one, and 

\item update the variable by writing back the incremented
value.  

\end{itemize}

This sequence is {\em non-atomic}, i.e., it is subject to interruption
at any step.  If an interruption cannot affect the computation, then
there is no harm in this.  However, should another thread also attempt
to execute this same region of code, the program steps could become
interleaved or superimposed in a way that causes the variable to be
updated incorrectly. A sequence of program steps that must be executed
without interruption is called a {\em critical section}.

To prevent unwanted interleaving, the author of a multithreaded
program must identify each region of code that must be executed
atomically, and which therefore represents a critical section.  The
programmer then protects each atomic region from interruption by
enclosing the code in the Chez Scheme special form {\sf
critical-section}.  The body of {\sf critical-section} is executed
atomically, i.e., with interrupts disabled.  Upon exit from {\sf
critical-section}, interrupts are automatically reenabled.  Full
details of {\sf critical-section} may be found in \cite{CSSM}.

We have now covered the basic language elements and system primitives
required by the thread system.


%-* Previous Work

\section{Previous Work}

This report documents the current version of the {\sc swl} thread system.
The latest version builds on the experience gained and lessons learned
from implementing and using the earlier attempts, and corrects serious
design flaws inherent in those versions.

Given the intrinsic support for multiple dynamic execution contexts
provided by continuations, it is relatively straightforward to
construct a system that supports explicit concurrency within the
language. However, it is considerably more difficult to develop a
system in which all of the details of concurrency management have been
adequately addressed.  The earlier versions have faltered because of
these problems. This section reviews the difficulties we encountered
previously.

As will be discussed later, correct handling of continuation winders
poses some subtle problems. Earlier versions of the thread system
always invoked continuation winders on a thread context switch,
leading to subtle bugs related to reentrancy of {\sf yield}.

Some earlier versions of the thread system created new thread control
blocks for interrupted threads.  This ``migrating thread identity''
property was expedient for implemention, but non-intuitive and overly
complex for the user.  In the current system thread identity does not
migrate.

Earlier versions of the system did not distinguish between resettable
(server) and non-resettable (non-server) threads.  All threads were
considered to be resettable, which could erroneously cause multiple
threads to ``listen'' for input on the same port.

The implementation of queues in some earlier versions led to awkward
problems in computing the queue on which a specific thread was
enqueued.  In the current implementation, given an enqueued thread,
computing its queue is straightforward.

Earlier versions of the system provided debugger support by providing
a dedicated ``debugger thread'' attached to the console.  However,
this idea did not support multiple concurrently executing programming
environments.  (In the current system the debugger is available as an
ordinary procedure.)

Earlier versions of the system included a non-blocking input-output
subsystem.  The current version does not directly provide non-blocking
input-output (however, {\sc swl} implements non-blocking input-output ports
on top of the thread system).


%-* Current System Architecture

\section{Current System Architecture}

This section describes the architecture of the current version of the
{\sc swl} thread system.  The thread system extends Chez Scheme to provide
{\sc swl} system and application developers with independent concurrency via
multiple threads of control. 

The thread system combines a Scheme continuation that is computed in a
special kind of way, called a {\em thread continuation\/} or {\em tk},
together with an additional data structure called a {\em thread
control block\/} or {\em tcb}, to form a {\em thread}, or
independently schedulable continuation.  Each thread is a member of a
collection of one or more related threads called a {\em process}. A
process has its own associated data structure called a {\em process
control block\/} or {\em pcb}. Grouping threads into processes
facilitates the sharing of certain resources among the group.


%-** Continuations and Threads

\subsection{Continuations and Threads}

We now discuss the use of continuations within the thread system in
detail.  While all Scheme implementations provide regular continuation
objects, variations of regular continuations have been proposed from
time to time.  The thread system uses one such variant provided by
Chez Scheme, a {\em one-shot continuation}, and defines its own
variant, a {\em thread continuation}.
 
A one-shot continuation differs from a regular continuation in that it
may be invoked at most one time.  When a one-shot has been invoked, it
is said to be ``shot;'' it is an error to call a one-shot continuation
that has already been shot. (In addition, it is an error in Chez
Scheme to attempt to inspect a shot one-shot using the Chez Scheme
inspector.)  This restriction on the full generality of continuations
may be exploited to obtain a more efficient implementation.  Since the
thread system must obtain a fresh thread continuation object on each
context switch (i.e., on each call to {\sf yield}), it may use
one-shots in place of regular continuations.  A one-shot continuation
is obtained by calling {\sf call/1cc}.

However, there are two reasons why not all continuations, whether
regular or one-shot, are suitable for use as thread continuations.
First, since by definition each thread represents an independent
computation, a thread continuation has no parent computation to which
to return.  Therefore, a thread continuation must be made to
``terminate'' once the underlying computation is finished.  To explain
the second reason we must first introduce the {\sf dynamic-wind}
primitive and continuation winders.

Some Scheme systems, including Chez Scheme, provide a primitive
procedure known as {\sf dynamic-wind}, which the program author may
use to ``protect'' program state against changes in normal flow of
control, such as which may occur when a continuation is invoked. {\sf
dynamic-wind} takes three thunks as arguments, called {\sf in}, {\sf
body}, and {\sf out}. The three thunks are executed in order, and the
return value of the {\sf dynamic-wind} is the return value of the
body.  However, if the body throws to a continuation that was not
obtained (by {\sf call/cc} or {\sf call/1cc}) within the body, then
{\sf out} will be evaluated before the throw is performed. Finally, if
execution of the body is subsequently restored by a throw to a
continuation that was captured within the body, {\sf in} will be
evaluated before the throw.  Thus, the in and out thunks of {\sf
dynamic-wind} may be used to ``guard'' the body even in the presence
of continuation throws.

In Chez Scheme the in and out thunks established during a call to {\sf
dynamic-wind} are called {\em winders}, and the ordered collection of
pairs of winders established by all the active calls to {\sf
dynamic-wind} is automatically stored with the current continuation,
in a Scheme list.  When a new continuation object is obtained, the
current winders list is also associated with the new continuation
object.

We are now ready to define thread continuations.  A thread
continuation is a regular Scheme continuation whose winder list is
disjoint from the winder list of any other thread continuation.  More
precisely, no sublist of the continuation's winders is {\sf eq?} to
any sublist of winders of another thread continuation.\footnote{This
is a slight oversimplification. In reality, all thread continuations
share the single winder list of the non-threaded continuation that was
in effect when the thread system was bootstrapped.} This restriction
ensures that each thread represents a dynamic execution context that
is independent of any other thread, i.e., that there is no logical
nesting relationship between thread continuations.

Thus, on a call to {\sf yield} we might expect the following sequence
to occur: first, the entire set of the current thread continuation's
out winders is evaluated (since the winder lists of the two threads
are disjoint). Next, certain thread context variables are changed to
reflect the execution of a new thread.  Then, the entire set of the
new thread continuation's in winders is evaluated. Finally, the actual
throw to the new thread continuation is performed.

While evaluation during a context switch of all the winders of each
thread appears to be reasonable, in practice this creates significant
problems.  First, if the context switch is implicit and preemptive,
executing the winders may not be what the user intended.

Second, the Chez Scheme primitive {\sf critical-section} is
implemented in terms of {\sf dynamic-wind}, and there are cases in
which calling {\sf yield} from tail position within a {\sf
critical-section} is useful.  In this case {\sf yield} will be invoked
while a {\sf critical-section} is active. However, evaluating all of
the current thread's out thunks temporarily cancels any active {\sf
critical-section}.  If {\sf yield} does this before the current thread
has been fully suspended (and there are several steps in the
suspension process), it leaves the thread vulnerable to an interrupt
and possibly an untimely context switch.

While the problem with {\sf critical-section} alone could be overcome,
there is a related problem: {\sf dynamic-wind} permits arbitrary user
code to be run from a thread's winder list, as there is no restriction
on what the in and out thunks may do.  It is conceivable that a
program could call {\sf yield} in the context of:

\begin{itemize}

\item a call to {\sf dynamic-wind} in which an
out thunk is declared that must be run with interrupts enabled (so
that, e.g., garbage collection could occur as needed), and 

\item a
{\sf critical-section} form to prevent other threads from running until the
current thread is fully suspended. 
\end{itemize}

Unfortunately, both criteria cannot be satisfied simultaneously.  To
avoid problems of this kind we must place restrictions on the
execution of winders during a thread context switch.  Two
possibilities will be considered here.

First, we could inhibit the execution of all winders during a thread
context switch.  Winder execution would occur only on continuation
throws performed by the user, and never on throws to thread
continuations (which are never available to the user).  This is in
fact the approach taken in the current version of the thread system.
It is straightforward to implement, but has the undesirable property
of preventing certain code that depends on the Chez Scheme special
form {\sf fluid-let} from being reentrant, including certain parts of
Chez Scheme itself.  For example, in the current version of the thread
system, it is an error for more than one thread to use the Chez Scheme
compiler concurrently, because of the reentrancy problem caused by the
improper handling of {\sf fluid-let}.\footnote{It is probably also an
error to invoke the compiler from an interrupt handler for essentially
the same reason.  Control does not pass to an interrupt handler via a
continuation throw, so no winder lists are evaluated.}

Second, we could inhibit only the execution of user-defined winders
during a thread context switch.  A user-defined winder is one that is
declared by an explicit user program call to {\sf dynamic-wind}.  An
implicit call to {\sf dynamic-wind}, e.g., by expansion of a {\sf
critical-section} form, would not be inhibited.  This approach is
slightly more complex to implement, but has the desirable property of
preserving the correctness of the Chez Scheme {\sf fluid-let} form for
multiple threads.  However, this approach is not feasible at present
unless Chez Scheme is modified to support it.

The Chez Scheme system-mode procedure \$current-winders is used by the
thread system to obtain the list of winders associated with the
current continuation, and to assign a new list of winders to the
current continuation.

%-** Data Structures and Procedures}

\subsection{Data Structures and Procedures}

Several essential thread system data stuctures and procedures are 
introduced here.

\subsubsection{Data Structures}
\begin{description}

\item[\sf tcb] the thread control block, created for each thread.
Contains the thread continuation (except for the running thread),
scheduling priority and time quantum, queue pointers (except for 
the running thread), the thread's process, and other data.

Enqueued threads contain pointers to two other {\sf tcb} objects ({\sf
next} and {\sf previous}). Each queue contains an extra {\sf tcb}
object, i.e., a ``dummy'' thread, which is created to serve as the
queue header. The {\sf run} and {\sf sleep} queues are statically
defined by the thread system, and various other queues (e.g., {\sf
error} and {\sf completed}) are created by the system dynamically. All
other queues are defined by the user.

\item[\sf pcb] the process control block, created for each
process, or group of related threads.  Contains the per-process
parameters, process name, and other data.

\item[\sf qcb] the queue control block, or queue header, created for each
thread queue. Contains slots for holding message queue data, and other
data.  

\end{description}


\subsubsection{Procedures}
\begin{description}

\item[\sf yield] suspends execution of the current thread by placing it on
the run queue (or another queue if specified as an argument), and
starts execution of another runnable thread.

\item[\sf fork] accepts a thunk and creates a new runnable thread (see
{\sf thunk$\rightarrow$tk}) in the current process, returning the new
tcb.

\item[\sf fork-process] accepts a thunk and creates a new runnable
thread (see {\sf thunk$\rightarrow$tk}) and a new {\em process},
returning the new tcb.  The new process inherits the parameters
of the current process.  The essential difference between a thread and
a process is that a process stores {\em threaded parameters}, not a
thread.  (see {\em Threaded Parameters} below.)

\item[\sf thunk$\rightarrow$tk] accepts a thunk and returns a new
thread continuation.  When subsequently started by {\sf yield}, the
thread continuation first evaluates the thunk, and then suspends its
thread on a ``completed'' queue (from which it is never restarted).

\item[\sf tbreak] accepts a thread and modifies its thread
continuation to execute the thunk assigned to the threaded parameter
{\sf interrupt-handler} before throwing to the continuation.  (see
{\sf Threaded Parameters} below.) {\sf tbreak\/} is useful for
defining the keyboard break action.

\item[\sf clone-thread] accepts a thread and returns a new thread
containing many of the same data as in the original.  {\sf
clone-thread\/} is useful for capturing the state of a thread that has
called error or tbreak.

\item[\sf make-queue] accepts a name argument and returns a new queue
object: a ``dummy'' thread tcb that contains a queue header object, or
qcb.  Queues are used for several purposes: scheduling and suspension,
and message passing and synchronization.

\item[\sf enqueue] accepts a thread and a queue and places the thread
on the queue.

\item[\sf dequeue] accepts a non-empty queue (i.e., a tcb) and
returns the first thread enqueued there.

\end{description}



%-** parameters 

\subsection{Parameters}

Chez Scheme provides access to certain toplevel system values through
an interface called a {\em parameter}.  A parameter is a procedure of
zero or one arguments.  If called with no argument, the parameter
simply returns the underlying {\em parameter value}.  With one
argument, the parameter first checks that the argument is of the
appropriate type for the parameter.  If so, the parameter assigns the
argument as the new parameter value, otherwise it signals an error.

A number of standard Chez Scheme parameters determine properties of
the system that may affect the execution of any continuation.
Examples are {\sf current-output-port} and {\sf print-length}, which
affect how program output is written.  Therefore we must consider how
threads and parameters interact.

Since each thread represents an independent computation, each thread
may require its own parameter values during its execution. The thread
system provides for this by redefining all system parameters whose
values may vary from thread to thread.  We will call such
redefinitions {\em threaded parameters}.

A separate set of threaded parameters is stored in each thread's
process control block.  Therefore, all of the threads of a process
share one set of parameters. (In fact, this is the most significant
distinction between threads and processes.)

The current implementation of threaded parameters assumes that a
parameter stores its value in a local variable.  While this is true
for all Chez Scheme parameters that were originally defined using the
system primitive {\sf make-parameter}, the specification of parameters
does not require this.  In fact, the Chez Scheme parameters {\sf
current-input-port} and {\sf current-output-port} assign global
variables as a side-effect, and therefore these parameters must
receive special handling within {\sf yield}.

Certain parameters should not be associated with individual threads
and therefore are not redefined.  An example of such a non-threaded
parameter is {\sf collect-request-handler}. (See the appendix for more
details.)

The Chez Scheme special form {\sf parameterize} may be used with both
regular and threaded parameters. However, since {\sf parameterize} is
implemented in terms of {\sf dynamic-wind}, its side-effect will not be
undone if the thread yields.  This is relevant only if (1) the
parameter is non-threaded (hence global), or (2) the parameterizing
thread yields to another thread of the same process, since otherwise
the other thread will access a different set of per-process parameter
values.


%-** timer interrupts and thread scheduling

\subsection{Timer Interrupts and Thread Scheduling}

Thread continuations abstract multiple independent dynamic execution
contexts but do not specify how control is exchanged between different
contexts.  Threads may schedule themselves explicitly by calling {\sf
yield}.  However, to provide for implicit (or preemptive) scheduling
we must introduce an additional language construct.  Chez Scheme
contains one such construct: the {\em timer}.  The thread system uses
the timer to implement {\em timed preemption}.

The Chez Scheme timer is a clock that counts steps, called {\em
ticks}, of an running program and interrupts the program when the
requested number of ticks have elapsed.  As a program is executed, one
tick is subtracted from the timer clock for each countable program
step. Not all program steps are counted by the system, however, and
not all steps take an equal amount of real time. Also, the body of a
{\sf critical-section} form is not charged for number of the ticks it
consumes.  These facts affect the accuracy of real time scheduling
decisions.

When the timer count reaches zero, the system performs a {\em timer
interrupt}. A timer interrupt is a forced, preemptive procedure call
to the {\em timer interrupt handler}, an ordinary Scheme thunk.

The user may enable, disable, and set the timer tick count with the
primitive procedure {\sf set-timer}.  When called with no arguments,
{\sf set-timer} returns the number of ticks left on the timer.  When
called with a non-negative fixnum integer argument, {\sf set-timer}
replaces the old tick value with the argument and returns the old tick
value.

The thread system manages the timer clock and the supplies the timer
interrupt handler.  It is an error for the user to alter these
objects.  Instead, the user may specify how a thread is to be
scheduled by supplying a {\em time quantum\/} for the thread.  A time
quantum is specified in units of {\em cpu time}, not ticks.  For
example, at present the default quantum is ten milliseconds of cpu
time, which means that the system will permit a thread to run without
yielding for up to ten milliseconds of elapsed cpu time (as charged to
the running Scheme task by the underlying operating system).

A thread's time quantum is an approximation of the actual cpu time
consumed.  Chez Scheme counts ticks, not time, so the thread system
must determine the latter indirectly.  This fact may render the thread
system unable to cope with severe real time scheduling constraints.

The thread system sets a single, fixed tick value into the timer.
When a timer interrupt is taken, the handler computes the amount of
cpu time elapsed since the previous interrupt, all of which is charged
to the currently running thread, and decides whether the current
thread should be forced to yield.

The accuracy of scheduling decisions varies inversely with the fixed
timer tick value established by the system. This value is set to a
default of 2000.  The default value was selected as a compromise
between scheduling accuracy and the overhead imposed by calling the
timer interrupt handler.  It is possible that another value may be
more appropriate under certain circumstances.

The thread system timer also provides a way for the user to perform
high priority, ``foreground'' processing tasks by setting the {\em
timer interrupt hook}, a Scheme thunk.  The thread system timer
interrupt handler calls the hook once each timer interrupt.

%-** keyboard interrupts

\subsection{Keyboard Interrupts}

The thread system has been designed for {\sc swl}, an interactive software
development and execution evironment.  It is expected that from time
to time the user will intervene to suspend or abort running
computations.  The thread system adapts the Chez Scheme keyboard
interrupt mechanism to work in a multi-threaded environment.

When a Chez Scheme task is started by the operating system, a {\em
console input port} object is allocated to it, which the thread system
assumes is attached to a terminal keyboard.  If the operating system
detects that the user has typed an {\em interrupt key sequence} at the
keyboard, it will signal this fact to Chez Scheme.  Chez Scheme in
turn signals the thread system by invoking the {\em keyboard interrupt
handler}, a Scheme thunk.

The thread system manages keyboard interrupts and supplies the
keyboard interrupt handler.  It is an error for the user to alter this
object.  When invoked, the keyboard interrupt handler signals a
distinguished thread, the {\em console}, by calling {\sf tbreak} with
the console as an argument. In addition, if a thread other than the
console is executing, the handler also signals that thread by calling
tbreak.

When given a keyboard interrupt, a running console will enter the
debugger.  Otherwise the console and running threads are simply reset.


%-** garbage collection, collect-request-handler

\subsection{Garbage Collection}

Like the timer and keyboard interfaces, the Chez Scheme garbage
collector is parameterized by a handler.  Each time a collection is
required, the system calls the {\em collect request handler}, a Scheme
thunk.  Unlike the timer and keyboard handlers, however, the collect
request handler is not set by the thread system, so the user is free
to supply one.

Because collections may occur at almost any time (except within the
body of a {\sf critical-section}), and because the user-specified collect
request handler may execute arbitrary code, the thread system must
consider the effects of invoking the handler.

A crucial issue is that the collect request handler may cause a thread
context switch.  This is most likely to be the case if the handler
writes to a non-blocking output port (such ports typically cause one
or more thread context switches on each operation).

To avoid problems caused by thread context switches triggered by any
interrupt handler, we design the {\sf yield} primitive to be {\em
reentrant}, i.e., capable of being interrupted and invoked again by
the interrupt handler.


%-** thread synchronization

\subsection{Message Passing and Synchronization}

While some computations are ``single-threaded'' and may therefore be
expressed as a single independent thread, other computations are best
expressed as a group of mutually cooperating threads.  While
cooperation between threads may be achieved without introducing new
primitives, because threads may update shared variables to exchange
information, and may busy-wait to synchronize their activities, these
techniques are error-prone and inefficient.  As discussed above in the
section on concurrency management, assignment of shared variables
generally introduces a critical section, which may be overlooked by
the user.  Furthermore, synchronization by busy-waiting wastes machine
cycles; it is more efficient to suspend the waiting thread until the
anticipated event occurs.

To overcome these limitations we introduce primitives that permit
a more abstract method of thread communication and synchronization
known as {\em message passing}.  Under message passing a thread may
forward data to another thread by {\em sending\/} a message to a
shared message {\em queue}, and it may synchronize its execution with
another thread by waiting to {\em receive\/} a message on a shared
queue.  We now describe the two thread system message passing
primitive procedures.

\begin{description}

\item[\sf send-msg] accepts a queue and a message (any Scheme object),
then either (1) places the message on the queue if there is no waiting
receiver thread, or (2) if a thread is waiting to receive a message,
sends the datum to the receiver by returning the thread to the run
queue with the message as the return value.

\item[\sf receive-msg] accepts a message queue and either (1) removes and
returns the next message from the queue if one is available, or (2)
enqueues the current thread on the message queue, thus causing it to
block until a message is available.

\end{description}


%-** breaking and resuming execution

\subsection{Breaking and Resuming Execution}


The keyboard interrupt, described earlier, is one mechanism for
diverting the normal flow of control of a thread.  In this section we
look more closely at the thread system constructs for managing such
diversions.

	tbreak primitive
	error handling
	server threads
	breaking servers, the console

%-** inspection and debugging

\subsection{Inspecting and Debugging}
	{\sf call/cc} vs. {\sf call/1cc}: can't inspect a shot continuation


%-* System Performance

\section{System Performance}

\begin{verbatim}

Environment: 200 MHz Pentium processor without secondary cache, Chez
Scheme 5.9b, optimize-level 2, best of five runs.


Benchmark code:

(define (fact n) (if (zero? n) 1 (* n (fact (1- n)))))
(define (loop c t) (if (zero? c) \#f (begin (t) (loop (1- c) t))))


Benchmark timings:

1. SWL 0.9l, no other running threads.

(time (loop 10000 (lambda () (fact 100))))
    34 collections
    3570 ms elapsed cpu time, including 10 ms collecting
    3561 ms elapsed real time, including 34 ms collecting
    36726320 bytes allocated, including 36732752 bytes reclaimed


2. SWL 0.9l, one additional running thread (which always yields 
immediately).

(time (loop 10000 (lambda () (fact 100))))
    34 collections
    3730 ms elapsed cpu time, including 40 ms collecting
    3733 ms elapsed real time, including 49 ms collecting
    37735272 bytes allocated, including 36855656 bytes reclaimed


3. No SWL or threads.

(time (loop 10000 (lambda () (fact 100))))
    34 collections
    3470 ms elapsed cpu time, including 10 ms collecting
    3463 ms elapsed real time, including 24 ms collecting
    36645432 bytes allocated, including 36841432 bytes reclaimed


ratio 1:3 =  1.03
ratio 2:3 =  1.07

\end{verbatim}

%-* Areas for Future Work

\section{Areas for Future Work}

%-** Desired Changes to Chez Scheme

\section{Proposed Extensions for Support of Concurrency}

This work has suggested several language implementation changes or
extensions that would benefit the thread system.

\begin{itemize}

\item We have found that the Chez Scheme {\sf critical-section} form
is rather expensive when used with extremely small critical sections,
such as code to increment a shared integer variable.  An atomic
read-increment-and-assign primitive, which could be used without an
enclosing {\sf critical-section}, would be useful.

\item Code that is protected by {\sf critical-section} is not charged
for its execution by decrementing the timer tick counter.  This leads
to scheduling inaccuracy whenever a large proportion of a thread's
code executes within the body of a critical section.  We suggest that
the body of a critical section should be charged for its execution.

\item It would be useful to distinguish between the dynamic-wind
procedure called by system code and in the expansions of system forms,
and the dynamic-wind that is available to the user.  This would permit
us to distinguish winders that are defined by system primitives and
forms from user-defined winders.  In this case we could restore the
full functionality of {\sf fluid-let}.


\end{itemize}



%-* Summary and Conclusion

\section{Summary and Conclusion}




%-* Acknowledgements

\section{Acknowledgements}

Carl Bruggeman implemented the first iterations of the thread system
and devised many of the concepts presented here.  Oscar Waddell, chief
architect of {\sc swl}, collaborated closely with Carl and the author.
R. Kent Dybvig supervised the overall {\sc swl} project and is the author of
Chez Scheme.


%-* Bibliography

\begin{thebibliography}{99}

\bibitem{TSPL} R. Kent Dybvig. {\em The Scheme Programming Language},
Second Edition, Prentice-Hall, Inc., Upper Saddle River, New Jersey,
1996.

\bibitem{CSSM} R. Kent Dybvig.  {\em Chez Scheme System Manual,
Revision 2.4}.  Cadence Research Systems, Bloomington, Indiana. 1994.

\bibitem{TCLTK} Ousterhout, J. {\em Tcl/Tk}, Sun Microsystems
Laboratories.

\end{thebibliography}




\end{document}



%-* Appendix: User's Guide

\section{Appendix: User's Guide and Reference Manual}


%-** Public Procedures

\subsection{Public Procedures}

thread-error-handler

A procedure called by the error primitive in response to a thread
execution error, which prints error data and thread-identifying
information to the console output port.


thread-timer-interrupt-hook

A thunk that is called once for each interrupt taken by the Chez
Scheme timer-interrupt.  Since the thread system triggers the
timer-interrupt at a very high frequency, it is essential that the
timer interrupt hook procedure be made as inexpensive as possible.


thread-yield

Forces the current thread to relinquish control to another runnable
thread.  The current thread's scheduling priority is reset to the
priority of its process (see thread-reschedule), and the thread is
placed in the run-queue according to its (new) priority.

The return value of thread-yield is used internally by the thread
system (by the message queue primitives), and should be ignored by
user code.


thread-sleep-check

thread-become-console!

thread-become-server!

thread-kill

thread-trav-run-queue

thread-quantum-remaining

thread-fork-process

thread-fork

thread-sleep

thread-reschedule

thread-make-msg-queue

thread-send-msg

thread-receive-msg

thread-msg-waiting?

thread-receiver-waiting?

thread->k

thread-find

thread-self

thread-number

thread-make-parameter

thread-debug

thread-break

ps

ps-all

ps-num

pps


%-** Syntax

\subsection{Syntax}

disable-timed-preemption
enable-timed-preemption


%-** Major Data Structures

\subsection{Data Structures}

The following data structures are defined using the Chez Scheme
{\sf define-structure} form.

A {\sf tcb}, or thread control block, is allocated for each thread.
Each tcb contains the following fields:

\begin{description}

\item[\sf num]	a unique integer identifier.

\item[\sf pri]	a integer, used to order running threads.

\item[\sf tk]	the thread continuation for suspended threads; undefined
		if the thread is running.

\item[\sf quantum]	the length of time in milliseconds the thread is
	permitted to run without preemption.

\item[\sf sleep] the length of time in milliseconds the thread is
to sleep; undefined if the thread is not sleeping.

\item[\sf pcb]		the thread's process control block.

\item[\sf next]		the next thread in the queue if the thread
is enqueued, else \#f.

\item[\sf prev]		the next thread in the queue if the thread
is enqueued, else \#f.

\item[\sf tpc]		the timed preemption counter (currently unused).

\item[\sf sic]		the system interrupt counter.

\item[\sf starts]	the number of times the thread has been started.

\item[\sf winders]	the thread's winder list; undefined if the 
thread is running.

\end{description}

A {\sf pcb}, or process control block, is allocated for each process, or
group of threads.  Each {\sf pcb} contains the following fields:

\begin{description}

\item[\sf tpri]
\item[\sf params]
\item[\sf threads]
\item[\sf shot?]
\item[\sf name]

\end{description}


A {\sf qcb}, or queue control block, is allocated for each queue.  Each
{\sf qcb} contains the following fields:

\begin{description}

\item[\sf info]
\item[\sf data]
\item[\sf rdata]
\item[\sf sends]

\end{description}




%-** Global Variables



thread-run-queue-idle?

\#t if one or more threads are ready to run, \#f otherwise.

thread-sleep-queue-idle?

\#t is one or more threads are sleeping, i.e., blocked on the 
sleep queue, \#f otherwise.


thread-conout

The original Chez Scheme console output port, a blocking port.



thread-highest-priority

A fixnum value that denotes the highest priority at which a thread
may be scheduled to run. (Larger numbers represent lower priorities.)


thread-lowest-priority

A fixnum value that denotes the lowest priority at which a thread may
be scheduled to run. (Larger numbers represent lower priorities.)




%-** Redefinitions of standard Chez Scheme Procedures

error

reset

new-cafe



%-** Global Parameters

thread-default-quantum

thread-default-ticks



%-** Thread-specific Parameters

last-error

cafe-level

interrupt-handler

process-server

process-name

thread-quantum


%-** effects on standard language constructs
	{\sf call/cc}, {\sf dynamic-wind}, {\sf fluid-let}, 
	{\sf parameterize}

%-** dependencies on Chez Scheme internals



	\$current-winders
	\$current-stack-link
	\$null-continuation
	\#scheme-version
	\#interrupt-flags
	subset-mode 




